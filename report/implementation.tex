
We have divided the program into four modules. Let us walk through each module's functionality & purpose.\\

Module 1: Loading the Data \\
This module serves two purposes.\\
First, its \emph{load\_data} function extracts the data from one of the two datasets
and returns two matrices containing the examples and labels. \\
Second, its \emph{get\_binary\_targets} function maps a matrix of labels to a binary target for a chosen emotion,
such that a '1' indicates that a label represents this emotion and a '0' indicates otherwise.

Module 2: Decision Tree Structure \\
We feel it is important to discuss our implementation of the decision tree structure,
because many extra functionalities are coded into it. The decision tree is made up of two types of objects.\\
First, the DecisionTree object with members:\\
attribute\_to\_test represents the attribute this node tests. Kids contains the two children nodes, one for
examples that have a '0' for this node's test and one for examples that have a '1'. Lastly, it has an id member used for graph
generation purposes. DecisionTree has an evaluate method which given an example, tests its attribute and recurses into
the appropriate kid's evaluate method. \\
Second, the LeafNode object which is a subclass of DecisionTree and thus inherits all its field, in addition to a new
member representing the class label of this leaf node, a '0' or a '1' representing the decision of the tree.
It overrides the evaluate method of the Decision tree as there is no recursion at a leaf node, it simply returns its class label.




Part 3: Training Decision Trees
Let us start now turn to our \emph{choose\_best\_decision\_attribute} implementation. \\
First, when seeking the best attribute to split the current node's examples on, we do not maximize the information gain.
Instead, we minimize the information remainder (i.e. the average entropy of the two children nodes
that would be created if splitting on a particular attribute).
This method yields the same results yet
improves the algorithm's complexity, as we are not repetitively recomputing the initial entropy of the node.\\
Second, we noticed that deeper down the tree, multiple attributes may be candidates to be the best attribute on a certain node,
because they offer the same information gain. We were faced with a dilemma: Should we simply
take the first one of those candidate attributes? We decided to randomize a selection from the candidates.

Next, let us take a closer look at the implementation \emph{find\_majority\_value}.\\
What value should it return when the given binary targets contains an equal number of
'1's and '0's? We decided to randomize the selection.

Now, let us discuss how we performed cross-validation\\
Here is the general process we follow:
split the data randomly into 10 folds.




2) cross validation perform



3) how you compute average results
4) how to test python code

