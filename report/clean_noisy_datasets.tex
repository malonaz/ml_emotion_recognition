TITLE: Noisy-Clean Datasets Question


It's clear we can identify performance diferrences between clean and noisy datasets.
Due to the fact that some data and attributes are missing in the noisy set, decision trees have more
difficulties to identify emotion and particurlarly anger and sadness.
Classification rate is dependant on the data quality and the algorithm used, 
and since the algorithm is fixed, we know that the noisy dataset has not truly
captured the actual attributes needed to distinguish between the emotions.


We can compare the performance for each emotion using clean and noisy dataset and
F1 value. In fact, it takes a weighted average of both recall rate and precision value. 


Anger in the clean set has an F1 value of 64.66\%, where as in the noisy dataset this 
value reduces significantly to 27.96\%, which implies that the accuracy almost halves 
when using the noisy data. 

All the other emotions remain F1 values which are proportionnal to the clean dataset. 
Because of the lack of some missing attributes, it's normal to find reduction of the
F1 value for all the emotions.

Fear's F1 value has reduced from 64.17\% to 54.35\%. 
Happiness's F1 value has also reduced from 82.12\% to 68.60\%.
Sadness from 53.30\% to 38.56\%. 
Surprise from 81.20\% to 72.05\%.

Surprise emotion seems to be easier one to identify with a very low decrease of F1 value
with noisy data.  

These values are expected since the reduction is not too great and given the 
fact that noisy data is not completely accurate.