We have (regrettably) decided to implement our code in Python and as such, provide you with information on how to use our code.
If you run our code on a linux computer in the lab, there are no dependencies. If you run it on your computer, you must install \colorbox{lightgray}{\emph{dot}}, to generate graphs.
We have implemented a makefile to make your life easier and will proceed to explain the various commands available to you.

 \colorbox{lightgray}{\emph{make all}} will run all the following make.\\
 \colorbox{lightgray}{\emph{make main}} runs our main file  \colorbox{lightgray}{src/main.py}. Results can be found
in the  \colorbox{lightgray}{outputs} folder.\\
 \colorbox{lightgray}{\emph{make graphs}} will call  \colorbox{lightgray}{src/main.py} which will train trees, and generate png graphs of the decision trees, for vizualization purposes,
for both clean and noisy datasets. You will find these in the  \colorbox{lightgray}{graphs} folder.
Those are directly compiled in our latex report.\\
 \colorbox{lightgray}{\emph{make report}} will compile the report, which will be located in  \colorbox{lightgray}{report/report.pdf}.\\


Note that our python main must always be called from the root folder, as filepaths are coded for such purpose.
We have prepared a template  \colorbox{lightgray}{src/test.py} file, that you may use when
trying to test our trees on new data.
Please use  \colorbox{lightgray}{\emph{make test}} to run it.




