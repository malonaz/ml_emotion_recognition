We have (regrettably) decided to implement our code in Python and as such, provide you with information on how to use our code.
If you run our code on a linux computer in the lab, there are no dependencies. If you run it on your computer, you must install \emph{dot}, to generate graphs.
We have implemented a makefile to make your life easier and will proceed to explain the various commands available to you.

\emph{make all} will run all the following make.\\
\emph{make main} runs our main file src/main.py.\\
\emph{make graphs} will call src/main.py which will train trees, and generate png graphs of the decision trees, for vizualization purposes,
for both clean and noisy datasets. You will find these in the graphs folder.
These are directly compiled in our latex report.\\
\emph{make report} will compile the report, which will be located in report/report.pdf.\\


Note that our python main must always be called from the root folder, as filepaths are coded for such purpose.
We have prepared a template main2.py file, that you may use when trying to test our trees on new data.
Please use \emph{make test} to run it.




